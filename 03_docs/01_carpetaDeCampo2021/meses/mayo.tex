\section{Mayo 2021}

\subsection*{Lunes 24/05/21}
    \begin{itemize}
        \item Reunion para ver estrategias Agiles, Scrum.
    \end{itemize}

\subsection*{Martes 25/05/21}
    \begin{itemize}
        \item Prueba Micropython en ESP32. \textcolor{red}{Descartado}, falta de soporte. Pudimos cargar los ejemplos de los programas basicis (blinky, leer puerto serie, i2c scan), pero vemos que hay pocas librerias desarrolladas para upython y upip.
            \begin{itemize}
                \item \href{https://micropython.org/download/}{uPython.}
                \item \href{http://docs.micropython.org/en/v1.15/esp32/tutorial/intro.html}{Tutorial uPython en ESP32.}
                \item \href{http://docs.micropython.org/en/v1.15/esp8266/tutorial/index.html#esp8266-tutorial}{Tutorial uPython en ESP8266.}
            \end{itemize}
        \item Se investiga sobre sensores usados en la industria
            \begin{itemize}
                \item Algunos de los mas usados son los de 'Meter Enviroment'
                \item Los modelos de la línea TEROS y ECH20 serían los mas adecuados para nuestra aplicación.
                \item Su costo va desde los U\$D 80 a U\$D 200.
            \end{itemize}
    \end{itemize}

\subsection*{Miércoles 26/05/21}
\begin{itemize}
    \item Prueba Libreria de sensor SHT (SHT-Esp32) en Heltec Lora ESP32 v2, \textcolor{red}{no funciona} en esta placa.
    \item Prueba Libreria de sensor SHT (SHT-Esp32) en placa Arduino Uno, \textcolor{ForestGreen}{funciona correctamente}.
        \begin{itemize}
            \item \href{https://github.com/beegee-tokyo/SHT1x-ESP}{Libreria SHT1x-ESP}
        \end{itemize}
    \item Prueba MQTT entre Heltec Lora-ESP32 y Raspberry pi 3b+ (funciona)
    \item Se busca información en general sobre mediciones de humedad, protocolos de calibración y tipos de sensores. Se encuentran dos documentos que describen los pasos correctos para la instalación de sensores y señalan diversos métodos de calibración.
        \begin{itemize}
            \item \href{https://www.metergroup.com/environment/events/soil-moisture-201-measurements-methods-and-applications/}{Webinario: Soil Moisture 201: Water Content Measurements, Methods, and Applications}
        \end{itemize}
    
\end{itemize}

\subsection*{Jueves 27/05/21}
\begin{itemize}
    \item Prueba Libreria de sensor SHT (SHT-Esp32) en NodeMCUv1 basado en ESP8266, \textcolor{ForestGreen}{funciona correctamente}.
        \begin{itemize}
            \item \href{https://github.com/beegee-tokyo/SHT1x-ESP}{Libreria SHT1x-ESP}
        \end{itemize}
    \item Prueba MQTT entre Heltec Lora-ESP32 y Raspberry pi 3b+ con envió de datos, fallas en puerto de com 1883 
    \item Se buscan fórmulas que nos permitan comprobar el correcto funcionamiento del sensor. Se encuentran muchas formas de medir la humedad, por lo que vamos a tener que probarlas respecto de nuestro sensor para ver cuál se adapta mejor.
    \item Se encuentra un seminario sobre mediciones de humedad en suelo.
    
\end{itemize}
\subsection*{Viernes 28/05/21}
\begin{itemize}
    \item Determinación de sensor SHT30 compra para actualizar el SHT11
    \item \textcolor{orange}{Compra 2 Sensores SHT30, MUX I2C TCA9548a}
\end{itemize}
\subsection*{Sábado 29/05/21 y Domingo 30/05/21}
    \begin{itemize}
        \item Día Libre.
    \end{itemize}

\subsection*{Lunes 31/05/21}
    \begin{itemize}
        \item Reunión para organizar la semana, hasta que vengan los sensores vamos a avanzar en el web server y en los procedimientos de calibración para los sensores.
        \item Prueba de MQTT + ESP32 (\textcolor{ForestGreen}{Andando!})
        \begin{itemize}
            \item \href{https://randomnerdtutorials.com/esp8266-and-node-red-with-mqtt/}{ESP8266, node-red, MQTT}
            \item \href{https://randomnerdtutorials.com/getting-started-with-node-red-on-raspberry-pi/}{Node-RED y RPi}
            \item \href{https://programarfacil.com/esp8266/mqtt-esp8266-raspberry-pi/}{MQTT, ESP8266, RPi}
        \end{itemize}
    \end{itemize}

\clearpage